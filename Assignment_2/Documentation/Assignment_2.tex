\documentclass[12pt, letterpaper]{article}
\usepackage[utf8]{inputenc}
\usepackage{graphicx}
\graphicspath{ {.} }
\usepackage{wrapfig}
\usepackage{subfig}
 
\title{ASTP-720 Assignment 2}
\author{Brendan Drachler}
\date{\today}

\begin{document}

\section*{Problem 1}

My numerical calculus library is contained in {numerical\_calculus.py}.

\section*{Problem 2}

$M(r)$, $\frac{d M(r)}{dr}$, and $M_{enc}(r)$ are all shown below for varying values of $c$ and $v_{200}$. One obvious thing worth noting is that higher $c$ values have a drastic impact on the mass at larger radii. In the case of $c=15$, the mass at large radii falls off much quicker. 

\begin{figure}[ht]
\begin{tabular}{ccc}
\includegraphics[width=50mm]{{v200_150.0_c_8_m}.png} &   \includegraphics[width=50mm]{{v200_150.0_c_8_dmdr}.png} &  \includegraphics[width=50mm]{{v200_150.0_c_8_menc}.png} \\

\includegraphics[width=50mm]{{v200_150.0_c_15_m}.png} &   \includegraphics[width=50mm]{{v200_150.0_c_15_dmdr}.png} &  \includegraphics[width=50mm]{{v200_150.0_c_15_menc}.png} \\
\end{tabular}
\caption{$M(r)$, $\frac{d M(r)}{dr}$, and $M_{enc}(r)$ for $c=8,15$ and $v_{200} = 150 km/s$}
\end{figure}

\begin{figure}[ht]
\begin{tabular}{ccc}
\includegraphics[width=50mm]{{v200_200.0_c_8_m}.png} &   \includegraphics[width=50mm]{{v200_200.0_c_8_dmdr}.png} &  \includegraphics[width=50mm]{{v200_200.0_c_8_menc}.png} \\

\includegraphics[width=50mm]{{v200_200.0_c_15_m}.png} &   \includegraphics[width=50mm]{{v200_200.0_c_15_dmdr}.png} &  \includegraphics[width=50mm]{{v200_200.0_c_15_menc}.png} \\
\end{tabular}
\caption{$M(r)$, $\frac{d M(r)}{dr}$, and $M_{enc}(r)$ for $c=8,15$ and $v_{200} = 200 km/s$}
\end{figure}

\begin{figure}[ht]
\begin{tabular}{ccc}
\includegraphics[width=50mm]{{v200_250.0_c_8_m}.png} &   \includegraphics[width=50mm]{{v200_250.0_c_8_dmdr}.png} &  \includegraphics[width=50mm]{{v200_250.0_c_8_menc}.png} \\

\includegraphics[width=50mm]{{v200_250.0_c_15_m}.png} &   \includegraphics[width=50mm]{{v200_250.0_c_15_dmdr}.png} &  \includegraphics[width=50mm]{{v200_250.0_c_15_menc}.png} \\
\end{tabular}
\caption{$M(r)$, $\frac{d M(r)}{dr}$, and $M_{enc}(r)$ for $c=8,15$ and $v_{200} = 250 km/s$}
\end{figure}

I also used my code to calculate the mass of the dark matter halo, assuming the radial extent of the Milky Way's luminous matter is $15 \ kpc$ and the radius of the dark matter halo is $100 \ kpc$, to be $1.681 10^{12} M_\odot$. 

\section*{Problem 3}

The library of matrix functions can be found in {matrix.py}.

\section*{Problem 4}
The unittest features are stored in matrix\_test.py. Running the script will automatically test matrix.py.

\end{document}